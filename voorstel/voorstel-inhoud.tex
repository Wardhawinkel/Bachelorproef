%---------- Inleiding ---------------------------------------------------------

% TODO: Is dit voorstel gebaseerd op een paper van Research Methods die je
% vorig jaar hebt ingediend? Heb je daarbij eventueel samengewerkt met een
% andere student?
% Zo ja, haal dan de tekst hieronder uit commentaar en pas aan.

%\paragraph{Opmerking}

% Dit voorstel is gebaseerd op het onderzoeksvoorstel dat werd geschreven in het
% kader van het vak Research Methods dat ik (vorig/dit) academiejaar heb
% uitgewerkt (met medesturent VOORNAAM NAAM als mede-auteur).
% 

\section{Inleiding}%
\label{sec:inleiding}

Waarover zal je bachelorproef gaan? Introduceer het thema en zorg dat volgende zaken zeker duidelijk aanwezig zijn:

\begin{itemize}
  \item kaderen thema
Digitale veiligheid is een essentiële vaardigheid geworden in een samenleving waar technologische vooruitgang niet stilstaat.
  \item de doelgroep
Jonggepensioneerden bezitten vaak voldoende digitale basisvaardigheden maar hebben minder ervaring met moderne beveiligingstool. Zo blijken ze moeilijk te overtuigen van het nut van wachtwoordmanagers.
Uit onderzoek door Ray et al(2021) blijkt dat deze groep wachtwoordmanagers als complex, onveilig of verwarrend beschouwen, ondanks dat ze baat zouden hebben bij veilig en gestructureerd wachtwoordbeheer. 
  \item de probleemstelling en (centrale) onderzoeksvraag
Deze bachelorproef vertrekt vanuit de nood aan een veilige, webgebaseerde simulatieomgeving waarin jonggepensioneerden zonder risico kunnen leren werken met een wachtwoordmanager. Deze oefenomgeving moet gebruikers toelaten om handelingen uit te voeren zoals genereren van sterke wachtwoorden, opslaan van logins en het herkennen van beveiligingsprincipes. Dit allemaal zonder dat er echte accounts of persoonsgegevens gebruikt worden.
De centrale onderzoeksvraag luidt daarom: "In welke mate is een game-engine, zoals Unity, geschikt om een webgebaseerde simulatie te ontwikkelen waarmee jonggepensioneerden op een veilige en toegankelijke manier kunnen oefenen met het gebruik van een wachtwoordmanager?"
Om deze vraag te beantwoorden wordt onderzocht welke noden, beperkingen en verwachtingen jonggepensioneerden hebben bij het leren gebruiken van wachtwoordmanagers, welke technische uitdagingen er bestaan bij het ontwikkelen van webgebaseerde simulaties en in welke mate Unity hierin een geschikte oplossing kan bieden.
  \item de onderzoeksdoelstelling
De doelstelling van dit onderzoek bestaat uit twee delen. Enerzijds wordt onderzocht aan welke functionele, technische en toegankelijkheidsvereisten een leeromgeving voor jonggepensioneerden moet voldoen. Anderzijds wordt er een protoype ontwikkeld om de mogelijkheden en beperkingen van Unity voor dit soort educatieve toepassingen te evalueren. Het eindresultaat moet ontwikkelaars inzicht geven in de bruikbaarheid van een game-engine voor webgebaseerde simulatie en bijdragen aan de ontwikkeling van digitale leeromgevingen voor deze specifieke doelgroep.
\end{itemize}

Denk er aan: een typische bachelorproef is \textit{toegepast onderzoek}, wat betekent dat je start vanuit een concrete probleemsituatie in bedrijfscontext, een \textbf{casus}. Het is belangrijk om je onderwerp goed af te bakenen: je gaat voor die \textit{ene specifieke probleemsituatie} op zoek naar een goede oplossing, op basis van de huidige kennis in het vakgebied.

De doelgroep moet ook concreet en duidelijk zijn, dus geen algemene of vaag gedefinieerde groepen zoals \emph{bedrijven}, \emph{developers}, \emph{Vlamingen}, enz. Je richt je in elk geval op it-professionals, een bachelorproef is geen populariserende tekst. Eén specifiek bedrijf (die te maken hebben met een concrete probleemsituatie) is dus beter dan \emph{bedrijven} in het algemeen.

Formuleer duidelijk de onderzoeksvraag! De begeleiders lezen nog steeds te veel voorstellen waarin we geen onderzoeksvraag terugvinden.

Schrijf ook iets over de doelstelling. Wat zie je als het concrete eindresultaat van je onderzoek, naast de uitgeschreven scriptie? Is het een proof-of-concept, een rapport met aanbevelingen, \ldots Met welk eindresultaat kan je je bachelorproef als een succes beschouwen?

%---------- Stand van zaken ---------------------------------------------------

\section{Literatuurstudie}%
\label{sec:literatuurstudie}

Hier beschrijf je de \emph{state-of-the-art} rondom je gekozen onderzoeksdomein, d.w.z.\ een inleidende, doorlopende tekst over het onderzoeksdomein van je bachelorproef. Je steunt daarbij heel sterk op de professionele \emph{vakliteratuur}, en niet zozeer op populariserende teksten voor een breed publiek. Wat is de huidige stand van zaken in dit domein, en wat zijn nog eventuele open vragen (die misschien de aanleiding waren tot je onderzoeksvraag!)?

Je mag de titel van deze sectie ook aanpassen (literatuurstudie, stand van zaken, enz.). Zijn er al gelijkaardige onderzoeken gevoerd? Wat concluderen ze? Wat is het verschil met jouw onderzoek?

Verwijs bij elke introductie van een term of bewering over het domein naar de vakliteratuur, bijvoorbeeld~\autocite{Hykes2013}! Denk zeker goed na welke werken je refereert en waarom.

Draag zorg voor correcte literatuurverwijzingen! Een bronvermelding hoort thuis \emph{binnen} de zin waar je je op die bron baseert, dus niet er buiten! Maak meteen een verwijzing als je gebruik maakt van een bron. Doe dit dus \emph{niet} aan het einde van een lange paragraaf. Baseer nooit teveel aansluitende tekst op eenzelfde bron.

Als je informatie over bronnen verzamelt in JabRef, zorg er dan voor dat alle nodige info aanwezig is om de bron terug te vinden (zoals uitvoerig besproken in de lessen Research Methods).

% Voor literatuurverwijzingen zijn er twee belangrijke commando's:
% \autocite{KEY} => (Auteur, jaartal) Gebruik dit als de naam van de auteur
%   geen onderdeel is van de zin.
% \textcite{KEY} => Auteur (jaartal)  Gebruik dit als de auteursnaam wel een
%   functie heeft in de zin (bv. ``Uit onderzoek door Doll & Hill (1954) bleek
%   ...'')

Je mag deze sectie nog verder onderverdelen in subsecties als dit de structuur van de tekst kan verduidelijken.

%---------- Methodologie ------------------------------------------------------
\section{Methodologie}%
\label{sec:methodologie}

Hier beschrijf je hoe je van plan bent het onderzoek te voeren. Welke onderzoekstechniek ga je toepassen om elk van je onderzoeksvragen te beantwoorden? Gebruik je hiervoor literatuurstudie, interviews met belanghebbenden (bv.~voor requirements-analyse), experimenten, simulaties, vergelijkende studie, risico-analyse, PoC, \ldots?

Valt je onderwerp onder één van de typische soorten bachelorproeven die besproken zijn in de lessen Research Methods (bv.\ vergelijkende studie of risico-analyse)? Zorg er dan ook voor dat we duidelijk de verschillende stappen terug vinden die we verwachten in dit soort onderzoek!

Vermijd onderzoekstechnieken die geen objectieve, meetbare resultaten kunnen opleveren. Enquêtes, bijvoorbeeld, zijn voor een bachelorproef informatica meestal \textbf{niet geschikt}. De antwoorden zijn eerder meningen dan feiten en in de praktijk blijkt het ook bijzonder moeilijk om voldoende respondenten te vinden. Studenten die een enquête willen voeren, hebben meestal ook geen goede definitie van de populatie, waardoor ook niet kan aangetoond worden dat eventuele resultaten representatief zijn.

Uit dit onderdeel moet duidelijk naar voor komen dat je bachelorproef ook technisch voldoen\-de diepgang zal bevatten. Het zou niet kloppen als een bachelorproef informatica ook door bv.\ een student marketing zou kunnen uitgevoerd worden.

Je beschrijft ook al welke tools (hardware, software, diensten, \ldots) je denkt hiervoor te gebruiken of te ontwikkelen.

Probeer ook een tijdschatting te maken. Hoe lang zal je met elke fase van je onderzoek bezig zijn en wat zijn de concrete \emph{deliverables} in elke fase?
In een eerste fase wordt een literatuurstudie uitgevoerd om bestaande kennis in kaart te brengen. Hierbij wordt onderzocht hoe jonggepensioneerden omgaan met wachtwoordmanagers, welke drempels zij ervaren, welke risico's verbonden zijn aan het niet gebruiken van dergelijke beveiligingstools en welke toegankelijkheidsprincipes relevant zijn voor deze doelgroep. 
In een tweede fase wordt een requirements-analyse uitgevoerd. Hierbij worden bestaande leeromgevingen rond wachtwoordbeheer geanalyseerd om inzicht te krijgen in hun sterke punten, beperkingen en geschiktheid voor jonggepensioneerden. Op basis hiervan worden de noodzakelijke funtionele en niet-functionele requirements voor een webgebaseerde simulatie opgesteld. 
In een derde fase wordt Unity geëvalueerd op technische haalbaarheid. Hierbij wordt onderzocht in welke mate Unity geschikt is voor webgebaseerde simulatie, specifiek gericht op performance binnen een WebGL-omgeving, toegankelijkheidsmogelijkheden, beveiligingsaspecten en verwachte ontwikkeltijd.
In een vierde en laatste fase wordt er een proof-of-concept ontwikkeld. Deze simulatie, gebouwd in Unity met WebGL-export, bevat onder andere volgende functies: het aanmaken van fictieve accounts, genereren en opslaan van wachtwoorden, feedback- en bewustwordingsmechanismen en een beveiligde sandboxomgeving waarin geen persoonlijke gegevens gebruikt kunnen worden. Het prototype dient als demonstratie van de haalbaarheid en gebruikerservaring van een webgebaseerde simulatie.

Tijdsplanning:
Literatuurstudie      week 1-3    Hoofdstuk literatuur
Requirementsanalyse   week 3-5    Requirementsdocument
Technische evaluatie  week 5-7    Evaluatierapport Unity
Proof-of-concept      week 7-12   Werkend prototype


%---------- Verwachte resultaten ----------------------------------------------
\section{Verwacht resultaat, conclusie}%
\label{sec:verwachte_resultaten}

Hier beschrijf je welke resultaten je verwacht. Als je metingen en simulaties uitvoert, kan je hier al mock-ups maken van de grafieken samen met de verwachte conclusies. Benoem zeker al je assen en de onderdelen van de grafiek die je gaat gebruiken. Dit zorgt ervoor dat je concreet weet welk soort data je moet verzamelen en hoe je die moet meten.

Wat heeft de doelgroep van je onderzoek aan het resultaat? Op welke manier zorgt jouw bachelorproef voor een meerwaarde?

Hier beschrijf je wat je verwacht uit je onderzoek, met de motivatie waarom. Het is \textbf{niet} erg indien uit je onderzoek andere resultaten en conclusies vloeien dan dat je hier beschrijft: het is dan juist interessant om te onderzoeken waarom jouw hypothesen niet overeenkomen met de resultaten.

In deze bachelorproef wordt een proof-of-concept ontwikkeld van een webgebasseerde simulatie waarmee jonggepensioneerden op een veilige en laagdrempelige manier kunnen oefenen met het gebruik van een wachtwoordmanager. 
Daarnaast wordt een technische evaluatie van Unity uitgevoerd op basis van meetbare criteria, waaronder performantie binnen een WebGL-omgeving, toegankelijkheidsmogelijkheden en ontwikkelcomplexiteit.. 
Verwacht wordt dat Unity geschikt kan zijn om een realistische simulatieomgeving te creëren maar dat WebGL-performance beperkingen kan opleggen op vlak van laadtijden en prestaties. Ook wordt verwacht dat Unity minder toegankelijkheidsopties biedt dan traditionele webtechnologieën.
De meerwaarde van deze bachelorproef ligt in het in kaart brengen van de technische haalbaarheid van Unity voor webgebaseerde simulaties en het ontwikkelen van een prototype. De resultaten kunnen bijdragen aan het ontwikkelen van beter toegankelijke leermiddelen en een verhoogde bewustwording rond het veilig wachtwoordbeheer bij jonggepensioneerden.