%==============================================================================
% Sjabloon onderzoeksvoorstel bachproef
%==============================================================================
% Gebaseerd op document class `hogent-article'
% zie <https://github.com/HoGentTIN/latex-hogent-article>

% Voor een voorstel in het Engels: voeg de documentclass-optie [english] toe.
% Let op: kan enkel na toestemming van de bachelorproefcoördinator!
\documentclass{hogent-article}

% Invoegen bibliografiebestand
\addbibresource{voorstel.bib}

% Informatie over de opleiding, het vak en soort opdracht
\studyprogramme{Professionele bachelor toegepaste informatica}
\course{Bachelorproef}
\assignmenttype{Onderzoeksvoorstel}


\academicyear{2025-2026} % TODO: pas het academiejaar aan

% TODO: Werktitel
\title{Webgebaseerde simulatie voor jonggepensioneerden: Een onderzoek naar de geschiktheid van Unity voor het aanleren van het gebruik van wachtwoordmanager}

% TODO: Studentnaam en emailadres invullen
\author{Ward Hawinkel}
\email{ward.hawinkel@student.hogent.be}

% TODO: Geef de co-promotor op
\supervisor[Co-promotor]{T. Clauwaert (HOGENT, \href{mailto:thomas.Clauwaert@hogent.be}{thomas.Clauwaert@hogent.be})}

% Binnen welke specialisatierichting uit 3TI situeert dit onderzoek zich?
% Kies uit deze lijst:
%
% - Mobile \& Enterprise development
% - AI \& Data Engineering
% - Functional \& Business Analysis
% - System \& Network Administrator
% - Mainframe Expert
% - Als het onderzoek niet past binnen een van deze domeinen specifieer je deze
%   zelf
%
\specialisation{Interprofessioneel}
\keywords{wachtwoordmanager, Proof-of-concepts, Unity}

\begin{document}

\begin{abstract}
Digitale gebruikers beschikken over voldoende digitale basisvaardigheden, maar onderzoek toont aan dat zij moeilijk te overtuigen zijn om wachtwoordmanagers te gebruiken. Ondanks hun vertrouwdheid met computers ervaren zij drempels zoals wantrouwen, beperkte kennis van voordelen en onzekerheid over veiligheid en gebruiksgemak. Aangezien veilig wachtwoordbeheer noodzakelijk is om digitale risico’s te beperken, bestaat er nood aan een leeromgeving waarin gebruikers veilig kunnen oefenen met het gebruik van een wachtwoordmanager.

Deze bachelorproef onderzoekt in welke mate een game-engine, in het bijzonder Unity, geschikt is voor het ontwikkelen van een webgebaseerde simulatie waarin beginnende gebruikers op een toegankelijke en veilige manier kunnen kennismaken met de functionaliteit van wachtwoordmanagers. Het onderzoek vertrekt vanuit bestaande literatuur over de obstakels die zij ondervinden bij het gebruik van een wachtwoordmanager en welke risico's ze lopen bij het niet gebruiken hiervan. Op basis van deze inzichten worden de behoeften en gebruiksproblemen van de doelgroep geanalyseerd, samen met de technische en didactische vereisten voor een veilige oefenomgeving. 
Vervolgens wordt een proof-of-concept uitgewerkt, die kan worden ingezet als workshop, waarin gebruikers in een veilige sandbox omgeving kunnen oefenen met het gebruik van een wachtwoordmanager. In deze simulatie worden realistische scenario's gesimuleerd zonder gebruik te maken van echte persoonlijke gegevens. 

De geschiktheid van Unity wordt beoordeeld op vlak van toegankelijkheid, veiligheid, performance en bruikbaarheid in webomgevingen. De evaluatie richt zich op de ontwikkelbaarheid van dergelijke simulaties, de meerwaarde voor de doelgroep en de beperkingen van Unity, in vergelijking met alternatieven, zoals bestaande kennisclips of webtoepassingen. De resultaten bieden inzicht in de haalbaarheid en de pedagogische waarde van webgebaseerde simulaties voor het versterken van online veilig gedrag bij beginnende gebruikers zonder ervaring met wachtwoordmanagers.
\end{abstract}

\tableofcontents

% De hoofdtekst van het voorstel zit in een apart bestand, zodat het makkelijk
% kan opgenomen worden in de bijlagen van de bachelorproef zelf.
%---------- Inleiding ---------------------------------------------------------

\section{Inleiding}%
\label{sec:inleiding}

% Waarover zal je bachelorproef gaan? Introduceer het thema en zorg dat volgende zaken zeker duidelijk aanwezig zijn:

\begin{itemize}

Digitale veiligheid is een noodzakelijke vaardigheid geworden in een samenleving waar technologische vooruitgang niet stilstaat. Wachtwoordmanagers zijn een belangrijk hulpmiddel om sterke, unieke wachtwoorden te genereren en beheren, wat belangrijk is voor het beschermen van persoonlijke digitale gegevens. 
Uit een wereldwijde survey van \textcite{Bitwarden2022} blijkt dat 32\% van de respondenten wachtwoorden hergebruikt over meerdere sites, 55\% vertrouwen op geheugen voor wachtwoorden en slechts 34\% gebruiken een wachtwoordmanager \autocite{Bitwarden2022}. Hieruit blijkt dat er een aanzienlijke kloof bestaat tussen de kennis over veilig wachtwoordbeheer en het gebruik ervan.
Deze bevindingen onderstrepen de nood aan een veilige, webgebaseerde oefenomgeving ("sandbox") waarin beginnende gebruikers zonder risico kunnen leren werken met een wachtwoordmanager. Deze oefenomgeving moet gebruikers toelaten om handelingen uit te voeren zoals genereren van sterke wachtwoorden, opslaan van logins en het herkennen van beveiligingsprincipes. Dit allemaal zonder dat er echte accounts of persoonsgegevens gebruikt worden.
Deze bachelorproef richt zich op gebruikers zonder ervaring met wachtwoordmanagers. Het testpubliek bestaat uit een representatieve groep mensen uit het persoonlijke netwerk van de student (vrienden, kennissen, familie, medestudenten). Zij zullen gevraagd worden om de ontwikkelde simulatieomgeving uit te proberen, zodat observaties kunnen worden verzameld over de user experience en bruikbaarheid van de tool.
De centrale onderzoeksvraag luidt: "In welke mate is de game-engine Unity geschikt om een webgebaseerde simulatie te ontwikkelen waarmee beginnende gebruikers zonder ervaring met wachtwoordmanagers op een veilige en toegankelijke manier kunnen oefenen met het gebruik ervan?"
Deze hoofdvraag wordt opgesplitst in volgende deelonderzoeksvragen:

\begin{enumerate}
    % Gebruikersonderzoek
    \item Welke verschillende typen wachtwoordmanagers bestaan er?
    \item Welke drempels ervaren beginnende gebruikers bij het gebruik van wachtwoordmanagers?
    \item Welke begeleidende informatie en tutorials bieden bestaande wachtwoordmanagers aan voor beginnende gebruikers?
    \item Welke functionele, technische en didactische noden hebben beginnende gebruikers zonder ervaring met wachtwoordmanagers bij het aanleren van veilig wachtwoordbeheer?
    \item Welke informatieve functies (bv. wachtwoordsterkte-indicatoren, waarschuwingen, security check-ups) bieden bestaande wachtwoordmanagers, en hoe dragen deze bij aan inzicht en gedragsverandering bij beginners?
    
    % Analyse bestaande leeromgevingen
    \item Welke risico’s of beperkingen bestaan er in bestaande leeromgevingen voor wachtwoordbeheer?
    \item Welke functionele en niet-functionele requirements volgen hieruit voor een veilige webgebaseerde simulatieomgeving?
    
    % Technische haalbaarheid Unity
    \item In welke mate ondersteunt Unity (met WebGL-export) deze requirements op vlak van performantie, toegankelijkheid, UX en beveiliging?
    \item Welke bestaande frameworks, libraries of workflows bestaan er binnen Unity voor het bouwen van simulaties, digitale trainingsomgevingen of “digital twins”?
    \item In welke mate zijn er Unity-assets, design patterns of template-scenes beschikbaar die kunnen worden hergebruikt voor het bouwen van een “fake computer” met fictieve browser en websites?
    \item Kan de simulatie generiek uitbreidbaar worden naar andere soorten cybersecurity-workshops, of moet elk scenario volledig vanaf nul ontwikkeld worden?
    \item Welke ontwerpkeuzes zijn nodig om een werkend proof-of-concept te realiseren, en welke beperkingen komen hierbij naar voren?
\end{enumerate}

  \item de onderzoeksdoelstelling
De doelstelling van dit onderzoek bestaat uit twee delen. Enerzijds in kaart brengen aan welke functionele, technische en toegankelijkheidsvereisten een leeromgeving voor beginnende gebruikers moet voldoen. Anderzijds een proof-of-concept ontwikkelen en evalueren, waarmee de mogelijkheden en beperkingen van Unity voor dit soort educatieve toepassingen duidelijk worden gemaakt. Het eindresultaat moet ontwikkelaars inzicht geven in de haalbaarheid, gebruikerservaring en mogelijke beperkingen van dergelijke simulatie.
\end{itemize}

%Denk er aan: een typische bachelorproef is \textit{toegepast onderzoek}, wat betekent dat je start vanuit een concrete probleemsituatie in bedrijfscontext, een \textbf{casus}. Het is belangrijk om je onderwerp goed af te bakenen: je gaat voor die \textit{ene specifieke probleemsituatie} op zoek naar een goede oplossing, op basis van de huidige kennis in het vakgebied.

%De doelgroep moet ook concreet en duidelijk zijn, dus geen algemene of vaag gedefinieerde groepen zoals \emph{bedrijven}, \emph{developers}, \emph{Vlamingen}, enz. Je richt je in elk geval op it-professionals, een bachelorproef is geen populariserende tekst. Eén specifiek bedrijf (die te maken hebben met een concrete probleemsituatie) is dus beter dan \emph{bedrijven} in het algemeen.

%Formuleer duidelijk de onderzoeksvraag! De begeleiders lezen nog steeds te veel voorstellen waarin we geen onderzoeksvraag terugvinden.

%Schrijf ook iets over de doelstelling. Wat zie je als het concrete eindresultaat van je onderzoek, naast de uitgeschreven scriptie? Is het een proof-of-concept, een rapport met aanbevelingen, \ldots Met welk eindresultaat kan je je bachelorproef als een succes beschouwen?

%---------- Stand van zaken ---------------------------------------------------

\section{Literatuurstudie}%
\label{sec:literatuurstudie}

%Hier beschrijf je de \emph{state-of-the-art} rondom je gekozen onderzoeksdomein, d.w.z.\ een inleidende, doorlopende tekst over het onderzoeksdomein van je bachelorproef. Je steunt daarbij heel sterk op de professionele \emph{vakliteratuur}, en niet zozeer op populariserende teksten voor een breed publiek. Wat is de huidige stand van zaken in dit domein, en wat zijn nog eventuele open vragen (die misschien de aanleiding waren tot je onderzoeksvraag!)?
%Je mag de titel van deze sectie ook aanpassen (literatuurstudie, stand van zaken, enz.). Zijn er al gelijkaardige onderzoeken gevoerd? Wat concluderen ze? Wat is het verschil met jouw onderzoek?
%Verwijs bij elke introductie van een term of bewering over het domein naar de vakliteratuur, bijvoorbeeld~\autocite{Hykes2013}! Denk zeker goed na welke werken je refereert en waarom.
%Draag zorg voor correcte literatuurverwijzingen! Een bronvermelding hoort thuis \emph{binnen} de zin waar je je op die bron baseert, dus niet er buiten! Maak meteen een verwijzing als je gebruik maakt van een bron. Doe dit dus \emph{niet} aan het einde van een lange paragraaf. Baseer nooit teveel aansluitende tekst op eenzelfde bron.
%Als je informatie over bronnen verzamelt in JabRef, zorg er dan voor dat alle nodige info aanwezig is om de bron terug te vinden (zoals uitvoerig besproken in de lessen Research Methods).
% Voor literatuurverwijzingen zijn er twee belangrijke commando's:
% \autocite{KEY} => (Auteur, jaartal) Gebruik dit als de naam van de auteur
%   geen onderdeel is van de zin.
% \textcite{KEY} => Auteur (jaartal)  Gebruik dit als de auteursnaam wel een
%   functie heeft in de zin (bv. ``Uit onderzoek door Doll & Hill (1954) bleek
%   ...'')
%Je mag deze sectie nog verder onderverdelen in subsecties als dit de structuur van de tekst kan verduidelijken.

\section{Het belang van veilig wachtwoordbeheer}

Het belang van veilig wachtwoordbeheer wordt in de literatuur vaak benadrukt. Gebruikers die hun wachtwoorden hergebruiken over meerdere sites lopen een verhoogd risico op cyberaanvallen \autocite{Gaw2006}. Studies tonen bovendien aan dat het gebruik van wachtwoordmanagers niet langer een luxe is, maar een noodzaak voor het behouden van digitale veiligheid \autocite{CMU2025}.

\section{Typen wachtwoordmanagers}
\textcite{Zawalnyski2025} verdeelt wachtwoordmanagers in drie categorieën: browser-, cloud- en desktop-gebaseerd. Verschillende internet browsers zoals Google Chrome en Safari, bieden ingebouwde wachtwoordmanagers aan die gebruikers in staat stellen om wachtwoorden op te slaan en automatisch in te vullen. Cloud-gebaseerrde wachtwoordmanagers worden meestal gedownload en beheerd op mobiele apparaten of desktop. Dit type slaan wachtwoorden op in een beveiligde cloudomgeving. Desktop-gebasseerde wachtwoordmanagers worden lokaal op de computer geïnstalleerd en slaan wachtwoorden lokaal op. 
Elk type heeft zijn voor- en nadelen, afhankelijk van de behoeften van de gebruiker.

\section{Drempels voor beginnende gebruikers}
Ondanks de voordelen blijkt uit onderzoek dat veel beginnende gebruikers terughoudend zijn om deze tools te gebruiken. Volgens \textcite{Pearman2019} hebben gebruikers vaak onvoldoende vertrouwen in de beveiliging van deze tools en ervaren de werking ervan als complex. Daarnaast weten sommige gebruikers niet dat er betere opties bestaan dan het simpelweg onthouden van wachtwoorden of het opschrijven ervan.  

\section{Beperkingen van traditionele leermethoden}
Binnen security awareness wordt steeds duidelijker dat traditionele methoden, zoals tekstuele handleidingen of tutorials, niet voldoende zijn om gebruikers effectief te trainen in veilig wachtwoordbeheer. Een belangrijke oorzaak is dat gebruikers wel advies krijgen over veilig gedrag, maar dit niet in de praktijk kunnen brengen zonder risico op fouten of datalekken \autocite{Sugatan2020}. 

\section{Interactieve leeromgevingen als oplossing}
\textcite{Sugatan2020} gebruikten in hun onderzoek een interactieve, verhalende leeromgeving waar gebruikers de kans kregen om beslissingen te nemen rond wachtwoordgedrag en vervolgens ook de gevolgen te zien. De onderzoekers stelden vast dat deze aanpak leidde tot een beter begrip, meer vertrouwen en een grotere bereidheid om zelf een wachtwoordmanager te gebruiken.
 Dit onderzoek onderstreept het potentieel van interactieve simulaties als een effectieve methode om beginnende gebruikers te trainen in veilig wachtwoordbeheer.

 

%---------- Methodologie ------------------------------------------------------
\section{Methodologie}%
\label{sec:methodologie}

%Hier beschrijf je hoe je van plan bent het onderzoek te voeren. Welke onderzoekstechniek ga je toepassen om elk van je onderzoeksvragen te beantwoorden? Gebruik je hiervoor literatuurstudie, interviews met belanghebbenden (bv.~voor requirements-analyse), experimenten, simulaties, vergelijkende studie, risico-analyse, PoC, \ldots?
%Valt je onderwerp onder één van de typische soorten bachelorproeven die besproken zijn in de lessen Research Methods (bv.\ vergelijkende studie of risico-analyse)? Zorg er dan ook voor dat we duidelijk de verschillende stappen terug vinden die we verwachten in dit soort onderzoek!
%Vermijd onderzoekstechnieken die geen objectieve, meetbare resultaten kunnen opleveren. Enquêtes, bijvoorbeeld, zijn voor een bachelorproef informatica meestal \textbf{niet geschikt}. De antwoorden zijn eerder meningen dan feiten en in de praktijk blijkt het ook bijzonder moeilijk om voldoende respondenten te vinden. Studenten die een enquête willen voeren, hebben meestal ook geen goede definitie van de populatie, waardoor ook niet kan aangetoond worden dat eventuele resultaten representatief zijn.
%Uit dit onderdeel moet duidelijk naar voor komen dat je bachelorproef ook technisch voldoen\-de diepgang zal bevatten. Het zou niet kloppen als een bachelorproef informatica ook door bv.\ een student marketing zou kunnen uitgevoerd worden.
%Je beschrijft ook al welke tools (hardware, software, diensten, \ldots) je denkt hiervoor te gebruiken of te ontwikkelen.
%Probeer ook een tijdschatting te maken. Hoe lang zal je met elke fase van je onderzoek bezig zijn en wat zijn de concrete \emph{deliverables} in elke fase?

Het onderzoek wordt uitgevoerd in vier fasen, elk gericht op het beantwoorden van specifieke deelonderzoeksvragen en het realiseren van de doelstellingen van de bachelorproef

\begin{enumerate}
    \item \textbf{Literatuurstudie (week 1-3):} \\
    In een eerste fase wordt een literatuurstudie uitgevoerd om bestaande kennis in kaart te brengen. Hierbij wordt onderzocht welke typen wachtwoordmanagers bestaan en welke info ze bieden, hoe beginnende gebruikers zonder ervaring omgaan met wachtwoordmanagers, welke drempels zij ervaren, welke risico's verbonden zijn aan het niet gebruiken van dergelijke beveiligingstools en welke toegankelijkheidsprincipes relevant zijn voor deze doelgroep. Deze inzichten vormen de basis voor de requirements-analyse.
    \item \textbf{Requirements-analyse (week 3-5):} \\
    In deze fase wordt een requirements-analyse uitgevoerd. Hierbij worden bestaande leeromgevingen rond wachtwoordbeheer geanalyseerd om inzicht te krijgen in hun sterke punten, beperkingen en geschiktheid voor beginnende gebruikers. Op basis hiervan worden de noodzakelijke funtionele en niet-functionele requirements voor een veilige webgebaseerde simulatie opgesteld. Deze requirements zullen als referentie dienen voor de proof-of-concept.
    \item \textbf{Technische evaluatie van Unity (week 5-7):} \\
    In een derde fase wordt Unity geëvalueerd op technische haalbaarheid. Hierbij wordt onderzocht in welke mate Unity geschikt is voor webgebaseerde simulatie, specifiek gericht op performance binnen een WebGL-omgeving, toegankelijkheid, UX, beveiliging en ontwikkeltijd. Tevens wordt gekeken naar bestaande frameworks, libraries en design patterns binnen Unity die relevant zijn voor het bouwen van een "fake computer" met fictieve browser en websites. 
    \item \textbf{Proof-of-concept ontwikkeling (week 7-12):} \\
    In een vierde fase wordt er een proof-of-concept ontwikkeld. Deze simulatie, gebouwd in Unity met WebGL-export, bevat onder andere volgende functies:
    \begin{itemize}
        \item het aanmaken van fictieve accounts en het genereren en opslaan van wachtwoorden;
        \item feedback- en bewustwordingsmechanismen die gebruikers begeleiden bij veilig wachtwoordbeheer;
        \item een beveiligde sandboxomgeving waarin geen persoonlijke gegevens gebruikt worden.
    \end{itemize} 
    Het proof-of-concept wordt getest door een kleine, representatieve groep beginnende gebruikers uit het netwerk van de student. het doel is om feedback te verzamelen over de gebruikservaring, bruikbaarheid en eventuele technische beperkingen van de simulatie.
  \end{enumerate}

\textbf{Deliverables per fase:}
\begin{itemize}
    \item Literatuurstudie: overzicht van bestaande kennis, uitdagingen en best practices voor wachtwoordmanagers en educatieve simulaties
    \item Requirements-analyse: document met functionele en niet-functionele requirements voor de webgebaseerde simulatieomgeving
    \item Technische evaluatie: rapport met haalbaarheid, beperkingen en aanbevelingen voor Unity 
    \item Proof-of-concept: werkend webgebaseerd prototype, inclusief observatierapport van gebruikerservaring
\end{itemize}


%---------- Verwachte resultaten ----------------------------------------------
\section{Verwacht resultaat, conclusie}%
\label{sec:verwachte_resultaten}
%Hier beschrijf je welke resultaten je verwacht. Als je metingen en simulaties uitvoert, kan je hier al mock-ups maken van de grafieken samen met de verwachte conclusies. Benoem zeker al je assen en de onderdelen van de grafiek die je gaat gebruiken. Dit zorgt ervoor dat je concreet weet welk soort data je moet verzamelen en hoe je die moet meten.
%Wat heeft de doelgroep van je onderzoek aan het resultaat? Op welke manier zorgt jouw bachelorproef voor een meerwaarde?
%Hier beschrijf je wat je verwacht uit je onderzoek, met de motivatie waarom. Het is \textbf{niet} erg indien uit je onderzoek andere resultaten en conclusies vloeien dan dat je hier beschrijft: het is dan juist interessant om te onderzoeken waarom jouw hypothesen niet overeenkomen met de resultaten.

In deze bachelorproef wordt een proof-of-concept ontwikkeld van een webgebasseerde simulatie waarmee beginnende gebruikers zonder ervaring met wachtwoordmanagers op een veilige en laagdrempelige manier kunnen oefenen met het gebruik van een wachtwoordmanager. 
Daarnaast wordt een technische evaluatie van Unity uitgevoerd op basis van meetbare criteria, waaronder performantie binnen een WebGL-omgeving, toegankelijkheid, gebruiksvriendelijkheid en ontwikkelcomplexiteit. 
Verwacht wordt dat Unity geschikt is om een realistische simulatieomgeving te creëren maar dat WebGL-performance beperkingen kan opleggen op vlak van laadtijden en prestaties. Ook wordt verwacht dat Unity minder toegankelijkheidsopties biedt dan traditionele webtechnologieën.

De meerwaarde van deze bachelorproef ligt in:
\begin{itemize} 
  \item het in kaart brengen van de technische haalbaarheid van Unity voor webgebaseerde educatieve simulaties; 
  \item het ontwikkelen van een proof-of-concept dat aanbiedt hoe beginnende gebruikers veilig kunnen leren werken met wachtwoordmanagers in een risicovrije omgeving;
  \item De resultaten kunnen bijdragen aan het ontwikkelen van beter toegankelijke leermiddelen en een verhoogde digitale bewustwording veilig wachtwoordbeheer.
\end{itemize}



\printbibliography[heading=bibintoc]

\end{document}